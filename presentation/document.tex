%%This is a very basic article template.
%%There is just one section and two subsections.
\PassOptionsToPackage{unicode}{hyperref}
\PassOptionsToPackage{naturalnames}{hyperref}

\documentclass{beamer}

\usetheme{Darmstadt}
%\usetheme[numbers, totalnumbers, minimal, nologo]{Statmod}

\usepackage[utf8]{inputenc}
\usepackage[T2A]{fontenc}
\usepackage[russian]{babel}
\usepackage{graphicx}
\usepackage{hyperref}
%\usepackage{lmodern}
%\usepackage[linesnumbered,boxed]{algorithm2e}
%\usepackage{algorithm}
%\userpackage{algpseudocode}
%\usepackage[format=plain, labelsep=period, justification=centerlast]{caption}
%\usepackage[centerlast]{caption2}
%\renewcommand{\captionlabeldelim}{{.}}

\let\Tiny=\tiny

\addtobeamertemplate{navigation symbols}{}{%
    \usebeamerfont{footline}%
    \usebeamercolor[fg]{footline}%
    \hspace{1em}%
    \insertframenumber/\inserttotalframenumber
}

\setbeamercolor{footline}{fg=black}
\setbeamerfont{Verbatim}{series=\bfseries}

\title
{\textbf{Моделирование критических свойств спиновых систем параллельным кластерным методом}}

\author
{\textbf{Ложников Вячеслав Евгеньевич}\texorpdfstring{\\Научный руководитель: д.ф.-м.н., проф. Прудников П.В.}{}}

\institute
{
\inst{}
Кафедра теоретической физики\\
ОмГУ им. Ф.М. Достоевского}

\date
{2014}

\begin{document}

\begin{frame}
  \titlepage
\end{frame}

%\setbeamertemplate{footline}[frame number]

\begin{frame}{\textbf{Цели и задачи}}
\begin{itemize}
    \item Исследование проблемы критического замедления при моделировании
    фазовых превращений численными методами Монте-Карло
    \item Вычисление динамического критического индекса $z$ для слабо
    неупорядоченных систем
    \item Создание параллельной реализации однокластерного алгоритма Вольфа
    \item Сравнительный анализ эффективности параллельной реализации с
    последовательной версией
\end{itemize}
\end{frame}

\begin{frame}{\textbf{Проблема критического замедления\footnotemark[1]}}
Общий случай:
%\begin{equation}
%    \tau \sim \xi^z
%\end{equation}
\begin{equation}
	%\xi \sim \left( (T - T_c)/T_c \right)^{-v}
	\tau_{corr} \sim \left( T - T_c \right)^{-vz}
\end{equation}
Для конечноразмерных систем:
\begin{equation}
	\tau \sim L^z
\end{equation}
\\
%$\xi$ - корреляционная длина
\\
\textit{z} - динамический критический индекс
\footnotetext[1]{Камилов И.К, Муртазаев А.К, Алиев Х.К., Исследование фазовых переходов и
критических явлений методами Монте-Карло.
// УФН. -- 1999. -- Т. 169. -- Вып. 7. -- C. 773–795.}
\end{frame}

\begin{frame}{Сравнительная характеристика алгоритмов по эффективности
преодоления проблемы критического замедления} 
\begin{table}\caption{Значение
динамического критического индекса \textit{z} для различных алгоритмов
применительно к однородной модели Изинга\footnotemark[2] }\footnotetext[2] {
Камилов И.К, Муртазаев А.К, Алиев Х.К., Исследование фазовых переходов и
критических явлений методами Монте-Карло.
// УФН. -- 1999. -- Т. 169. -- Вып. 7. -- C. 773–795.}
\begin{center}
%\vspace{0.5cm}
%\hspace{3.5cm}
\begin{tabular}{|c|c|c|c|c}
\hline
   Algorithm & Model & $2d $ & $3d$ \\
   \hline
	Metropolis & Izing & $2.1$ & $2.0$ \\
	Swedsen-Wang & Izing & $0.35$ & $0.55$ \\ 
	Wolf & Izing & $ 0.35$ & $0.40$ \\
\hline
\end{tabular}
%\vspace{0.5cm}
\end{center}
\end{table}
\end{frame}

\begin{frame}{}
\begin{figure}[t]
$C\left(\tau\right) \propto \exp{^{-t/\tau}}$
    \includegraphics[width=0.6\textwidth]{zgraph1}
    \caption{Сравнительный анализ алгоритмов на примере автокорреляционной
    функции\footnotemark[3]}
\footnotetext[3]{{K. Rummukainen,
Simulation methods in physics. \\ http://www.helsinki.fi/~rummukai/lectures/}}
    %\includegraphics[width=0.4\textwidth]{zgraph1}
%\caption{Визуализация полученных результатов:}
	\label{fig:2}
\end{figure}

\end{frame}

\begin{frame}
\begin{figure}[h]
    \includegraphics[width=0.6\textwidth]{zgraph2}
    \caption{Время автокорелляции $\tau_{int}$
    для решеток различных размеров в точке фазового перехода для 2-мерной
    модели Изинга\footnotemark[3]}
    \footnotetext[3]{{K. Rummukainen,
Simulation methods in physics. \\ http://www.helsinki.fi/~rummukai/lectures/  }}
    %\includegraphics[width=0.4\textwidth]{zgraph1}
%\caption{Визуализация полученных результатов:}
	\label{fig:1}
\end{figure}
\end{frame}

\begin{frame}
\begin{table}\caption{Сравнительное время автокорреляции для различных
алгоритмов, время затрачиваемое на один шаг Монте-Карло\footnotemark[3]}
\footnotetext[3]{K.Rummukainen, Simulation methods in physics. \\
http://www.helsinki.fi/~rummukai/lectures/}
\begin{center}
%\vspace{0.5cm}
%\hspace{3.5cm}
\begin{tabular}{|c|c|c|c}
\hline
   Algorithm & $\tau_{int}$ & time(ms)/iteration \\
   \hline
	Cluster & $5.7$ & $0.28$ \\
	Metropolis & $271$ & $1.4$ \\ 
	HB & $316$ & $1.6$ \\
\hline
\end{tabular}
%\vspace{0.5cm}
\end{center}
\end{table}
\end{frame}

\begin{frame}{Однокластерный алгоритм Вольфа}
	\begin{enumerate}
	  \item Выбор произвольного спина $s_i$
	  \item Проверка соседних спинов. Если $s_j = s_i$  добавить $s_j$ в
	  кластер с вероятностью $P_{add} = 1 - \exp(-2/T)$  
	  \item Повторить шаг 2, если спин $s_j$ был добавлен в кластер. Повторять
	  данную процедуру пока растет кластер
	  \item При окончании формирования кластера перевернуть спины ему принадлежащие
	  с вероятностью $P = 1$
	  \end{enumerate}
\end{frame}

\begin{frame}{Пример реализации алгоритма Вольфа в псевдокоде с использованием
рекурсии }
	%\begin{algorithm}[H]
	%\SetAlgoLined
	\tiny
	\begin{verbatim}
	
	void update\_cluster()		\{	\\
		\hspace{1cm}int start, state;	\\
		\\
		\hspace{1cm}start = dran() * volume;	/* начальные координаты,  */		\\	
		\hspace{1cm}state = s[start];	/* начальное состояние спина */	\\
		\\
		\hspace{1cm}/* добавляем элементы в кластер и переворачиваем его */	\\
		\hspace{1cm}grow\_cluster(start, state); \\	\} \\
	void grow\_cluster(int loc,int state)	\{	\\
		\hspace{1cm}int dir,new\_loc;	\\
		\\
		\hspace{1cm}s[loc] = -s[loc];	\hspace{1cm}/* переворачиваем спин */	\\
		\\
		\hspace{1cm}/* проходим по ближайшим соседям */	\\
		\hspace{1cm}for (dir=0; dir<n\_neighbours; dir++) \{	\\
		\hspace{1.5cm}new\_loc = neighbour[dir][loc]; \\
		\hspace{1.5cm}/* с вероятностью $ 1 - \exp(-2*\beta)$ добавляем элемент в
		кластер */
		\\
		\hspace{1.5cm}if (s[new\_loc] == state \&\& exp(-2.0*beta) < dran())	\{ \\
		\hspace{2.5cm}     grow\_cluster(new\_loc,state);	\\ 
		\hspace{1.5cm}\}	\\
		\hspace{1cm}\}		\\
		\}		\\	     
	\end{verbatim}
		  
	%}
	%\end{algorithm}
\end{frame}

\begin{frame}{Методы ускорения вычислений}
	\begin{itemize}
	  \item Параллелизация кода при помощи MPI, OpenMP, phtreads
	  \item Реализация на GPU\footnotemark[4], использование
	  Cuda, OpenCL
	  \item Универсальная реализация с помощью HSA (Heterogeneous System
	  Architecture)
	\end{itemize}
	\footnotetext[4]{Block B.J. , Preis T., Computer simulations of the Izing model
	on GPU. // Eur.Phys. J. B. -- 2012 -- V. 210 -- P. 133-145. }
\end{frame}

\begin{frame}{Общая идея параллелизации}
	\begin{enumerate}
	  \item На начальном этапе формирования кластера (при добавлении первой цепочки
	  спинов), разделить ``волновой фронт'' между процессорами. Координаты всех
	  спинов хранятся в общей памяти
	  \item Вводится некоторое число N, которое является целым кратным количества
	  процессоров P, оптимизирующим параметром. Каждой
	  области ``волнового фронта'' сопоставляется P mod N вычислительных элементов,
	  т.е каждый процессор работает с некоторой частью волнового фронта (частью
	  массивов данных связанных с ним)
	  \item Формируется общий список координат последних добавленных в кластер
	  спинов. Это происходит таким образом, что каждый процессор
	  записывает свой список в определенное место общего массива, а главный
	  процессор ведет подсчет числа элементов добавленных в кластер.
	  \footnote[4]{J.
	  Kaupuzs J., Rimsans J., Melnik V.N., Parallelization of the Wolff
single-cluster algorithm. // Phys. Rev. E -- 2010 -- V. 81. -- P. 02670.}
	\end{enumerate}
\end{frame}

\begin{frame}{Основные требования при параллелизации различных алгоритмов}
	\begin{enumerate}
	  \item Примерно равный объем вычислений в выделяемых подзадачах
	  \item Минимальный информационный обмен данными ( если есть возможность
	  разделить исходные данные на непересекающиеся области - стоит это сделать )
	  \item Конечная цель - балансировка нагрузки т.е не допустить простоя
	  вычислительных элементов системы (это происходит в случае разделения на малое
	  количество подзадач) или избытка межпоточных коммуникаций ( в случае
	  разделения на большое количество параллельных подзадач, обычно много большее
	  или сопоставимое с количеством процессоров )
	\end{enumerate}
\end{frame}

\begin{frame}
\begin{figure}[h]
    \includegraphics[width=0.6\textwidth]{amdals_law}
    \caption{Закон Амдала, ограничение роста производительности системы с
    увеличением количества вычислителей}
    %\includegraphics[width=0.4\textwidth]{zgraph1}
%\caption{Визуализация полученных результатов:}
	\label{fig:3}
\end{figure}
\end{frame}

\begin{frame}{Значения динамического критического	\\
индекса Z для параллельного алгоритма Вольфа}
\begin{table}[h]
\caption{\centering В случае слабо неупорядоченной системы с P=0.8}
	\centering
	\begin{tabular}{|c|c|c|c|c|c|c|c|c|}
    	\hline 
        P & L & 8  & 16 & 24 & 32 & 48 & 64 & approx \\
        \hline
        0.8 & $z$ & 0.35 & 0.49 & 0.44 & 0.54 & 0.57 & 0.67 & 0.62 \\
        
        \hline
    \end{tabular}
\end{table}

\begin{table}[h]
\caption{\centering В случае идеальной системы}
	\centering
	\begin{tabular}{|c|c|c|c|c|c|c|c|c|}
    	\hline 
        P &L & 8  & 16 & 24 & 32 & 48 & 64 & approx  \\
        \hline
        1 & $z$ & 0.22 & 0.16 & 0.33 & 0.15 & 0.49 & 0.46 & 0.40  \\
        \hline
    \end{tabular}
\end{table}
\end{frame}

\begin{frame}
	\begin{figure}[h]
	\includegraphics[width=0.5\linewidth]{graphics/Z}
	\includegraphics[width=0.5\linewidth]{graphics/Z_def}
		\caption{\centering Индекс $z$ для параллельного алгоритма Вольфа:\\ a)
							Однородная система б) Неоднородная система с P=0.8}
		\label{ris:td_chars}
	\end{figure}
\end{frame}

\begin{frame}{Эффективность параллельной реализации одноклатерного алгоритма
Вольфа} \begin{table}[h]
\caption{Время выполнения параллельного и последовательного алгоритмов для
3х-мерной модели Изинга в диапазоне температур от 4 до 5 единиц обменного
интеграла, с шагом 0.1, и с уменьшением шага до 0.01 в диапазоне температур 4.4
- 4.6 }
\centering
    \label{tabular:comparetime}
	\begin{tabular}{|c|c|cccccc|}
	\hline
    Algorithm & L & mcs_{max} & stat & P & t & accel & perfomance \\
    \hline
    Wolf & 64 & 1000 & 20 & 2 & 2h:53m & 1.66  & 18/2 Gflops \\
    Wolf & 64 & 1000 & 20 & 1 & 4h49m  & 1  & 18/4 Gflops \\
    Wolf & 48 & 1000 & 20 & 2 & 0h:36m & 1.6  & 18/2 GFlops	\\
	Wolf & 48 & 1000 & 20 & 1 & 0h:59m & 1  & 18/4 GFlops\\
	Wolf & 32 & 1000 & 20 & 2 & 0h:31m & 1.6 & 18/2 GFlops\\
	Wolf & 32 & 1000 & 20 & 1 & 0h:49m & 1  & 18/4 GFlops\\
    \hline
\end{tabular}
\end{table}	
\end{frame}

\begin{frame}{Основные выводы}
\begin{enumerate}
  \item Кластерные алгоритмы - эффективный способ преодоления проблемы
  критического замедления
  \item Динамический критический индекс для слабо неупорядоченных систем ( $z =
  0.62$ ) выше, чем для идеальных ( $z = 0.40$ )
  \item Параллельная реализация алгоритма Вольфа показывает ускорение
  вычислений в 1.6 раз
\end{enumerate}
\end{frame}

\end{document}
